% Created 2024-07-28 Sun 16:34
% Intended LaTeX compiler: pdflatex
\documentclass[11pt]{article}
\usepackage[utf8]{inputenc}
\usepackage[T1]{fontenc}
\usepackage{graphicx}
\usepackage{longtable}
\usepackage{wrapfig}
\usepackage{rotating}
\usepackage[normalem]{ulem}
\usepackage{amsmath}
\usepackage{amssymb}
\usepackage{capt-of}
\usepackage{hyperref}
\author{jim}
\date{\today}
\title{}
\hypersetup{
 pdfauthor={jim},
 pdftitle={},
 pdfkeywords={},
 pdfsubject={},
 pdfcreator={Emacs 28.2 (Org mode 9.5.5)}, 
 pdflang={English}}
\begin{document}

\textbf{James Rothstein}
Kaiser: 53662095
DOB:  12/23/1955

\section{Purpose}
\label{sec:orgcc112e3}
\begin{enumerate}
\item To obtain regular colonoscopy at Kaiser's Portland facilitates.
\item To regain regular access to appropriate Kaiser specialists (specifically
urology, nephrology, GI and pancreas doctors), on a continual,
ongoing,  when necessary basis.
\end{enumerate}


\section{Background}
\label{sec:org9fda6ee}

\subsection{Colonoscopy}
\label{sec:org1828cef}
My mother died of colon cancer at age 63.  Since my late 40s, I have
had regular colonoscopies every 3-5 years.   It has now been 7 years since the last procedure.

Not an expert, but having done this 5 or 6 times I am able to note how
facilities differ in procedures, especially when a complication
arises. (I do not empty on the usual time scale, even with 7 day
liquid diets.  This is well-known, in the charts, discussed with doctors both
before and after.)


Though all prior colonoscopies were successfully, the most recent one was complete fiasco.
(Peace Health in Springfield, OR, March 2024).


For purposes of this document, I was left with no colonoscopy, no
doctors to ask and a colon full of prep fluid.  What followed included
failure to return phone calls, a pinky re-damaged (after 35 years!),
a visit to ER, and two weeks of diarrhea with a stretch to match.  There
was no excuse for this and it must never happen again: the doctors
completely abdicated any responsibility.

Yet, the insurance company/CCO said it
paid the bill and considers the matter "private."  This puts others at risk.

It is astounding to me that people responsible for health, safety and
financial outcomes see nothing wrong.

The upshot is the procedure, which still must be done, should be done
at Kaiser facilities where the doctors have already successfully
navigated this twice.



\subsection{Specialist Access}
\label{sec:org297b1e3}

I have been under Kaiser's care for kidney stones since 2014.
Following 2 URS procedures in early 2014, I have had small stones but no
major events (I appreciate!).

However, urine oxalate (UrOx) has always been high, sometimes extremely
high.  Through much effort over period of years, I worked through a
series of doctors (at Kaiser, OHSU, U Chicago, Mayo Clinic).  The
problem was traced to fat malabsorption; large amount of fecal fat and
pancreatic insufficiency. -

Surprisingly, I have no symptoms of any of this.  
Also, enzyme replacements showed no benefit and were stopped.

However, MRI revealed a large number of pancreatic cysts.

To his credit, Dr. Mueller, my primary care physician raised several
questions about possible long-term benefits of enzymes, and proper way
to monitor the pancreatic cysts, including possible biopsy.  He
asked for "chart review" with appropriate specialists.  This was
rejected.  This was the trigger for this document.

Although the "chart review" was rejected. I was authorized for repeat
MRI, not a biopsy.  Baffling.  The doctor who requested the repeat MRI
is out-of-the office till 2057, when I will 102.

When I called to make a GI appointment, there was no GI appointments
and told to call back another time.  I do not know if a future
appointment this will also be rejected.

In the absence of any contrary information, I will do the 2nd MRI,
scheduled for July 27, 2024.

All I ask for here is be in regular care of specialists, for guidance,
to avoid repeating the background over and over; to watch for any symptoms or
organ changes, and mostly to avoid suprises.
\end{document}